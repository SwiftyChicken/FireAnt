Dit document beschrijft het implementatie in fase 2 van het spel \texttt{Fire Ant} in scheme \textbf{r5rs}.
Het maakt deel uit van het opleidingsonderdeel "Programmeerproject 1".

\section{Spelbeschrijving}
In dit spel is het de bedoeling dat een speler een \textit{vuurmier} bestuurt,
die tracht zijn mierenkoningin te redden van schorpioenenleger.
Om dit te voltooien moet de speler een serie van levels, die toenemen in moeilijkheidsgraad,
oplossen zodat hij tot bij de koningin geraakt.
Een level bestaat uit een doolhof gevuld met puzzels, items en vijanden.
Enkel door deze puzzels op te lossen en de schorpioenen te vermijden, kan de speler het level oplossen.
Het spel eindigt dan tot dat alle levels opgelost zijn of tot dat de speler al zijn levens verloren heeft.

\section{Implementatie}
Voor het tweede fase werd het spel uitgebreid met meerdere nieuwe functionaliteiten.

Onder andere:
\begin{itemize}
	\item Een kan 2 soorten schorpioenen bevatten die elk op een verschillend manier bewegen.
		Deze schorpioenen kunnen ook tijdelijk versnellen, op een willekeurige moment.
	\item	Het doolhof kan ook deuren bevatten die open gemaakt moeten worden door een sleutel die de speler moet nemen.
	\item Het doolhof bevat overstroomt gebieden die de speler enkel kan overlopen met een surfplank.
	\item Het doolhof bevat verschillende voedsel objecten die elk een verschillende effect heeben op de score.
	\item Het doolhof bevat verschillende voedsel objecten die elk een verschillende effect heeben op de score.
	\item Het spel bevat ook een scorebord die enkele belangrijke informaties weergeeft.
		Zoals hoogste score, levens, sleutels, etc.
	\item Het spel bevat voeding dat levens terug geeft.
	\item Een doolhof kan kogels bevatten, die gebruikt kunnen worden om schorpioenen dood te schieten.
	\item Het spel bevat 3 verschillende levels met verschillende puzzels.
		Het Spel herstart als het laatste level bereikt wordt
\end{itemize}

Het programma werd verdeelt uit
een controller (zal tussen de modellen en views moeten communiceren)
modellen (bezitten al de spellogica)
en de views (zorgt voor het tekenlogica van het spel)

