In deze sectie zullen alle ADTs en hun proceduren beschreven worden 
die in het programma voorkomen.
Deze ADTs zullen verdeelt worden in 3 categorieën:
\nameref{controller},
\nameref{model} en
\nameref{view}
%%%%%%%%%%%%%%%%%%%%%%%%%%%%% CONTROLLER %%%%%%%%%%%%%%%%%%%%%%%%%%%%%%%%%%
\section{Controller}
\label{controller}
De controller zal centraal zijn om de interacties tussen de modellen en views te beheren. 

\subsection{Game ADT}
\label{section:game}
Het \texttt{\nameref{section:game}} is het centrale ADT dat het spel opstart en beheert.
Deze is verantwoordelijk voor
het initialiseren van de \texttt{\nameref{section:player}},
het initialiseren en starten van de juiste \texttt{\nameref{section:level}},
het verwerken van de speler invoer
en het behandelen van het spellus.

\begin{table}[hbt]
\centering
\begin{tabular}{|ll|}
\hline
\rowcolor[HTML]{000000} 
{\color[HTML]{FFFFFF} \textbf{Naam}} & {\color[HTML]{FFFFFF} \textbf{Signatuur}} \\ \hline
new-game    & $($list $\rightarrow$ Game$)$                  \\ \hline
start       & $(\varnothing \rightarrow \varnothing)$        \\ \hline
\end{tabular}
\caption{Operaties van het \texttt{\nameref{section:game}}}
\label{table:1}
\end{table}

Het \texttt{\nameref{section:game}} bevat de volgende operaties:

\begin{itemize}
	\item \textbf{new-game}: Deze operatie maakt een object van het \texttt{\nameref{section:game}} aan.
	\item \textbf{start}: Start een nieuw spel op.
		Eerst slaagt het alle namen van de levels (dit zijn csv bestanden) uit de "level" map op, in een lijst.
		Deze lijst gebruikt het om de eerste \texttt{\nameref{section:level}} object te maken.
		Vervolgens maakt het een nieuwe object van het \texttt{\nameref{section:player}} en het \texttt{\nameref{section:view}}.
		Tenslotte "start" het, de spellus en gebruikers invoer met behulp van de game-loop en key-handler procedures respectievelijk. 
\end{itemize}

%%%%%%%%%%%%%%%%%%%%%%%%%%%%%% MODELLEN %%%%%%%%%%%%%%%%%%%%%%%%%%%%%%%%%%%
\section{Modellen}
\label{model}
De Modellen zullen verantwoordelijk zijn voor het spellogica m.a.w. alle berekening achter de schermen.

\subsection{Level ADT}
\label{section:level}
Het \texttt{\nameref{section:level}} is het ADT dat een level opbouwt aan de hand van csv tekst bestand.
Deze is verantwoordelijk om
een csv bestand te lezen en zijn elementen ervan te interpreteren,
het onthouden en aanmaken van alle nodige objecten in de level (bv. schorpioenen, eiren, het doolhof, etc.),
de speler te laten herstarten,
het updaten van de elementen in de level,
het onthouden van de startpositie (spawn)
en te checken of het level is uitgespeeld.

\begin{table}[hbt]
\centering
\begin{tabular}{|ll|}
\hline
\rowcolor[HTML]{000000} 
{\color[HTML]{FFFFFF} \textbf{Naam}} & {\color[HTML]{FFFFFF} \textbf{Signatuur}} \\ \hline
new-level      & (Player string $\rightarrow$ Level)   \\ \hline
get-maze       & ($\varnothing$ $\rightarrow$ Maze)                        \\ \hline
get-items       & ($\varnothing$ $\rightarrow$ list)                        \\ \hline
get-doors       & ($\varnothing$ $\rightarrow$ list)                        \\ \hline
get-scorpions  & ($\varnothing$ $\rightarrow$ list)                        \\ \hline
get-updates    & ($\varnothing$ $\rightarrow$ list)                        \\ \hline
is-finished?   & (Player $\rightarrow$ boolean)                            \\ \hline
is-legal-move? & (Object\footnotemark{} symbol $\rightarrow$ boolean)      \\ \hline
update!        & ($\varnothing$ $\rightarrow$ $\varnothing)$               \\ \hline
kill-all!        & ($\varnothing$ $\rightarrow$ $\varnothing)$               \\ \hline
clear-updates!        & ($\varnothing$ $\rightarrow$ $\varnothing)$               \\ \hline
try-shooting!        & ($\varnothing$ $\rightarrow$ $\varnothing)$               \\ \hline
try-opening!        & ($\varnothing$ $\rightarrow$ $\varnothing)$               \\ \hline
try-surfing!        & ($\varnothing$ $\rightarrow$ $\varnothing)$               \\ \hline
respawn        & ($\varnothing$ $\rightarrow$ $\varnothing$)               \\ \hline
\end{tabular}
\caption{Operaties van het \texttt{\nameref{section:level}}}
\label{table:level}
\end{table}


Het \texttt{\nameref{section:level}} bevat de volgende operaties:
\begin{itemize}
	\item \textbf{new-level}: Neemt een \texttt{\nameref{section:level}} en de naam van level csv bestand op als argument op.
		Deze procedure geeft een object van het type level terug.
		Deze level object bezit alle elementen die het level bezit.
	\item \textbf{get-maze}: Geeft de \texttt{\nameref{section:maze}} object terug opgeslagen in de \texttt{\nameref{section:level}}.
	\item \textbf{get-items}: Geeft een lijst van \texttt{Items} objecten terug die opgeslagen zijn in de \texttt{\nameref{section:level}}.
	\item \textbf{get-items}: Geeft een lijst van \texttt{Doors} objecten terug die opgeslagen zijn in de \texttt{\nameref{section:level}}.
	\item \textbf{get-scorpions}: Geeft een lijst van \texttt{\nameref{section:scorpion}} objecten terug die opgeslagen zijn in de \texttt{\nameref{section:level}}.
	\item \textbf{get-updates}: Geeft een lijst van objecten terug die in de huidige iteratie van de spellus geüpdatet zijn.
	\item \textbf{is-finished?}: Geeft aan of het spel is uitgespeeld door te kijken of de \texttt{\nameref{section:player}} helemaal onderaan de doolhof is geraakt.
	\item \textbf{is-legal-move?}: Neemt een Object (dat een \texttt{\nameref{section:positie}} bezit) en een direction ('up, 'down, 'left or 'right) als argument op.
		Het kijk dan of het object die richting uit kan gaan.
	\item \textbf{update!}: Deze functie update alle nodige objecten die zich in het level bevinden.
	\item \textbf{kill-all!}: Deze functie zorgt er voor dat alle schorpioenen verwijderd worden in het volgende update iteratie.
	\item \textbf{clear-updates!}: Maakt de lijst leeg van de elementen die geüpdatet moeten worden
	\item \textbf{try-shooting!}: Kijkt of de speler een kogel kan schieten.
		Door een object van de type \texttt{Bullet} maken als de Speler genoeg ammo heeft.
	\item \textbf{try-opening!}: Kijkt of de speler tegen aan een deur loopt en of deze open gemaakt kan worden.
	\item \textbf{try-surfing!}: Kijkt of de speler tegen aan water loopt en of het deze kan oversteken.
	\item \textbf{respawn}: Deze procedure plaatst de Player terug op zijn start positie en geeft hem terug ``leven''.
\end{itemize}

\footnotetext{Object is een object dat een Positie ADT bezit.}

\subsection{Maze ADT}
\label{section:maze}
Het \texttt{\nameref{section:maze}} is verantwoordelijk om het spellogica van het doolhof te construeren.

\begin{table}[hbt]
\centering
\begin{tabular}{|ll|}
\hline
\rowcolor[HTML]{000000} 
{\color[HTML]{FFFFFF} \textbf{Naam}} & {\color[HTML]{FFFFFF} \textbf{Signatuur}} \\ \hline
new-maze   & ($\varnothing$ $\rightarrow$ Maze)                          \\ \hline
get-type   & ($\varnothing$ $\rightarrow$ symbol)                        \\ \hline
get-unit   & (number number $\rightarrow$ symbol)                        \\ \hline
clear-path! & (number number $\rightarrow$ $\varnothing$)                        \\ \hline
add!  & (number number symbol $\rightarrow$ $\varnothing$)                        \\ \hline
is-accessible?   & (number number $\rightarrow$ boolean)                       \\ \hline
get-height   & ($\varnothing$ $\rightarrow$ number)                        \\ \hline
get-width   & ($\varnothing$ $\rightarrow$ number)                        \\ \hline
\end{tabular}
\caption{Operaties van het \texttt{\nameref{section:maze}}}
\label{table:maze}
\end{table}

Het \texttt{\nameref{section:maze}} bevat de volgende operaties:

\begin{itemize}
	\item \textbf{new-maze}: Deze operatie maakt een object van het \texttt{\nameref{section:maze}} aan.
	\item \textbf{get-type}: Geeft het type van de ADT terug in dit geval 'maze.
	\item \textbf{get-unit}: Geeft het symbol terug dat op de meegegeven rij en kolom staat.
	\item \textbf{clear-path!}: Vervang de symbool, op de meegegeven positie, met 'empty.
	\item \textbf{add!}: Zet een symbol op de mee gegeven rij en kolom.
	\item \textbf{is-accessible?}: Checkt of de gegeven rij en kolom leeg ('empty) is en of de positie bestaat in het \texttt{\nameref{section:maze}}.
	\item \textbf{get-height}: Geeft het hoogte van de doolhof terug.
		Deze is gedefinieerd door het GRID-HEIGHT constante in het ``etc/constante.rkt'' bestand.
	\item \textbf{get-width}: Geeft het hoogte van de doolhof terug.
		Deze is gedefinieerd door het GRID-WIDTH constante in het ``etc/constante.rkt'' bestand.
\end{itemize}

\subsection{Position ADT}
\label{section:positie}

De \texttt{\nameref{section:positie}} is verantwoordelijk voor het bijhouden en aanpassen van de positie, oriëntatie en snelheid van een object.

\begin{table}[hbt]
\centering
\begin{tabular}{|ll|}
\hline
\rowcolor[HTML]{000000} 
{\color[HTML]{FFFFFF} \textbf{Naam}} & {\color[HTML]{FFFFFF} \textbf{Signatuur}} \\ \hline
new-position     & (number number $\rightarrow$ Positie)                       \\ \hline
get-type         & ($\varnothing$ $\rightarrow$ symbol)                        \\ \hline
get-x            & ($\varnothing$ $\rightarrow$ number)                        \\ \hline
get-y            & ($\varnothing$ $\rightarrow$ number)                        \\ \hline
get-orientation  & ($\varnothing$ $\rightarrow$ symbol $\cup\{\#f\}$)          \\ \hline
get-speed        & ($\varnothing$ $\rightarrow$ number)                        \\ \hline
set-x!           & (number $\rightarrow$ $\varnothing$)                        \\ \hline
set-y!           & (number $\rightarrow$ $\varnothing$)                        \\ \hline
set-moving!      & (bool $\rightarrow$ $\varnothing$)                          \\ \hline
set-orientation! & (symbol $\rightarrow$ $\varnothing$)                        \\ \hline
set-speed!       & (number $\rightarrow$ $\varnothing$)                        \\ \hline
reset-speed!     & ($\varnothing$ $\rightarrow$ $\varnothing$)                 \\ \hline
is-moving?       & ($\varnothing$ $\rightarrow$ boolean)                       \\ \hline
is-colliding?    & (Position $\rightarrow$ boolean)                            \\ \hline
move!            & (symbol $\rightarrow$ $\varnothing$)                        \\ \hline
peek             & (symbol $\rightarrow$ Position)                             \\ \hline
copy             & ($\varnothing$ $\rightarrow$ Position)                      \\ \hline
\end{tabular}
\caption{Operaties van het \texttt{\nameref{section:positie}}}
\label{table:positie}
\end{table}

Het \texttt{\nameref{section:positie}} bevat de volgende operaties:

\begin{itemize}
	\item \textbf{new-positie}: Deze operatie maakt een nieuw object van het \texttt{\nameref{section:positie}} aan. 
		Deze neemt 2 nummers op als argument
		dat respectievelijk de x- en y-co\"ordinaat representeren in het \texttt{\nameref{section:maze}}.
	\item \textbf{get-type}: Geeft het type van de ADT terug in dit geval 'position.
	\item \textbf{get-x}: Geeft x positie terug.
	\item \textbf{get-y}: Geeft y positie terug.
	\item \textbf{get-orientation}: Geeft de oriëntatie van het object terug.
		Het is standaard \#f.
	\item \textbf{get-speed}: Geeft de snelheid van het object terug.
		De snelheid in dit context is 
		de snelheid waarmee het spel een tile van 1 positie naar een ander tekent.
		Deze is standaard 0,17.
	\item \textbf{set-x}: Veranderd de x positie door de gegeven nummer.
	\item \textbf{set-y}: Veranderd de y positie door de gegeven nummer.
	\item \textbf{set-moving}: Veranderd boolean waarde van het moving variabel.
		De variabel geeft aan of het object nog steeds aan het bewegen is van zijn vorige positie naar zijn nieuwe positie.
	\item \textbf{set-orientation}: Veranderd de oriëntatie positie door de gegeven richting.
	\item \textbf{set-speed!}: Veranderd de snelheid waarmee het object zich tussen twee posities verplaatst.
	\item \textbf{reset-speed!}: Veranderd de snelheid naar de standard snelheid waarden.
	\item \textbf{is-moving?}: Geeft aan of het object zich nog steeds aan het verplaatsen is naar zijn nieuwe positie.
	\item \textbf{is-colliding?}: Checkt of het meegegeven Positie de zelfde positie heeft in de Maze.
	\item \textbf{move!}: Verandert het positie en oriëntatie van het \texttt{\nameref{section:positie}} in de gegeven richting.
		Zolang deze positie nog niet aan het bewegen is naar een nieuw positie.
	\item \textbf{peek}: Geef de positie terug van buur positie gelegen op de meegegeven richting.
	\item \textbf{copy}: maakt en geeft een copy van het huidige positie object.
\end{itemize}

\subsection{Item ADT}
\label{section:item}

Het \texttt{\nameref{section:item}} stelt een voorwerp voor dat kan opgeraapt  worden door een \texttt{Player}.

\begin{table}[hbt]
\centering
\begin{tabular}{|ll|}
\hline
\rowcolor[HTML]{000000} 
{\color[HTML]{FFFFFF} \textbf{Naam}} & {\color[HTML]{FFFFFF} \textbf{Signatuur}} \\ \hline
new-item                              & (Positie $\rightarrow$ Item) \\ \hline
get-position         & ($\varnothing$ $\rightarrow$ Position)                        \\ \hline
is-taken?                         & ($\varnothing$ $\rightarrow$ boolean)               \\ \hline
take!                              & ($\varnothing$ $\rightarrow$ $\varnothing$)                   \\ \hline
\end{tabular}
\caption{Operaties van het \texttt{\nameref{section:item}}}
\label{table:item}
\end{table}

Het \texttt{\nameref{section:item}} bevat de volgende operaties:

\begin{itemize}
	\item \textbf{new-item}: Maakt een nieuw object van het \texttt{\nameref{section:item}}.
	\item \textbf{get-position}: Geeft het \texttt{\nameref{section:positie}} object terug die het egg object bezit.
	\item \textbf{is-taken?}: Check of het \texttt{\nameref{section:item}} is opgeraapt of niet.
	\item \textbf{take!}: Veranderd taken waarden naar \#t.
		Met andere worden het egg is opgerapen.
\end{itemize}

\subsection{Player ADT}
\label{section:player}
De \texttt{\nameref{section:player}} stelt het speler voor.
Deze ADT is verantwoordelijk om de conditie van de speler bij te houden.
Met andere woorden zijn aantal levens, items, zijn positie en de huidige leven status.

\begin{table}[hbt]
\centering
\begin{tabular}{|ll|}
\hline
\rowcolor[HTML]{000000} 
{\color[HTML]{FFFFFF} \textbf{Naam}} & {\color[HTML]{FFFFFF} \textbf{Signatuur}} \\ \hline
new-player    & ($\varnothing$ $\rightarrow$ Player) \\ \hline
get-type      & ($\varnothing$ $\rightarrow$ symbol)                        \\ \hline
get-position  & ($\varnothing$ $\rightarrow$ Position $\cup\{\#f\}$)                        \\ \hline
get-lives      & ($\varnothing$ $\rightarrow$ number)                        \\ \hline
get-keys      & ($\varnothing$ $\rightarrow$ number)                        \\ \hline
get-boards      & ($\varnothing$ $\rightarrow$ number)                        \\ \hline
get-ammo      & ($\varnothing$ $\rightarrow$ number)                        \\ \hline
get-points      & ($\varnothing$ $\rightarrow$ number)                        \\ \hline
set-position! & (Position $\rightarrow$ $\varnothing$)                        \\ \hline
is-dead?      & ($\varnothing$ $\rightarrow$ boolean)               \\ \hline
is-changed?      & ($\varnothing$ $\rightarrow$ boolean)               \\ \hline
use!          & (symbol $\rightarrow$ $\varnothing$)                   \\ \hline
collect!          & (symbol number $\rightarrow$ $\varnothing$)                   \\ \hline
reset!          & ($\varnothing$ $\rightarrow$ $\varnothing$)                   \\ \hline
die!          & ($\varnothing$ $\rightarrow$ $\varnothing$)                   \\ \hline
revive!       & ($\varnothing$ $\rightarrow$ $\varnothing$)                   \\ \hline
update!       & ($\varnothing$ $\rightarrow$ $\varnothing$)                   \\ \hline
\end{tabular}
\caption{Operaties van het \texttt{\nameref{section:player}}}
\label{table:ant}
\end{table}

De \texttt{\nameref{section:player}} bevat de volgende operaties:

\begin{itemize}
	\item \textbf{new-player}: Maakt een object van \texttt{\nameref{section:player}} aan.
	\item \textbf{get-type}: Geeft het type van de ADT terug in dit geval 'player.
	\item \textbf{get-position}: Geeft de \texttt{\nameref{section:positie}} object terug die het player object bezit.
	\item \textbf{get-lives}: Geeft aantal levens terug die de Player bezit.
	\item \textbf{get-keys}: Geeft het aantal Keys terug die de Player bezit.
	\item \textbf{get-boards}: Geeft het aantal Boards (surfplanken) die de speler bezit.
	\item \textbf{get-ammo}: Geeft het aantal Ammo terug die de Player bezit.
	\item \textbf{get-points}: Geeft de huidige score van de Player terug.
	\item \textbf{set-position!}: Geeft de \texttt{\nameref{section:player}} een nieuw \texttt{\nameref{section:positie}} object.
	\item \textbf{is-dead?}: Checkt of de \texttt{\nameref{section:player}} dood is of niet.
	\item \textbf{is-changed?}: Checked of de locale variabelen van Player zijn veranderd.
	\item \textbf{collect!}: Verhoogd de gevraagde locale variabele van Player met de meegegeven hoeveelheid.
	\item \textbf{use!}: Verminderd de gevraagde locale variabele met 1.
	\item \textbf{die!}: Veranderd de ``alive'' waarden naar \#f
		en trekt 1 leven van lives af.
		Met andere woorden speler is gestorven.
	\item \textbf{revive!}: Veranderd de alive waarden naar \#t als de speler nog lives heeft.
		De oriëntatie wordt ook gezet naar 'down omdat dit de standaard start oriëntatie is.
	\item \textbf{update!}: Geeft aan dat de Player niet veranderd is door de variabele ``changed'' naar \#f te veranderen.
\end{itemize}

\subsection{Scorpion ADT}
\label{section:scorpion}
De \texttt{\nameref{section:scorpion}} is de ADT verantwoordelijk voor het spellogica van de schorpioen. 
Deze ADT is verantwoordelijk om de snelheid en kleur van de schorpioen aan te passen.

\begin{table}[hbt]
\centering
\begin{tabular}{|ll|}
\hline
\rowcolor[HTML]{000000} 
{\color[HTML]{FFFFFF} \textbf{Naam}} & {\color[HTML]{FFFFFF} \textbf{Signatuur}} \\ \hline
new-scorpion    & (symbol Positie list $\cup$ symbol $\rightarrow$ Scorpion) \\ \hline
get-type      & ($\varnothing$ $\rightarrow$ symbol)                        \\ \hline
get-position  & ($\varnothing$ $\rightarrow$ Position)                        \\ \hline
get-color  & ($\varnothing$ $\rightarrow$ symbol)                        \\ \hline
is-alive?  & ($\varnothing$ $\rightarrow$ bool)                        \\ \hline
try-boosting!                    & (number $\rightarrow$ $\varnothing$)                                       \\ \hline
die!                                 & ($\varnothing$ $\rightarrow$ $\varnothing$)                                       \\ \hline
\end{tabular}
\caption{Operaties van het \texttt{\nameref{section:scorpion}}}
\label{table:scorpion}
\end{table}

Het \texttt{\nameref{section:scorpion}} bevat de volgende operaties:

\begin{itemize}
	\item \textbf{new-scorpion}: Maakt een nieuwe object aan van het \texttt{\nameref{section:scorpion}}.
		Deze object neemt als argument symbol, dat een kleur aan geeft,
		een Positie (die bewaard wordt)
		en een lijst van argumenten of een symbol dat door de Scorpion-Green of Scorpion-Yellow wordt gebruikt.
	\item \textbf{get-type}: Geeft het type van de ADT terug in dit geval 'scorpion.
	\item \textbf{get-position}: Geeft de \texttt{\nameref{section:positie}} object terug die het schorpioen object bezit.
	\item \textbf{get-color}: Geeft het kleur (symbol) van de Scorpion terug
	\item \textbf{is-alive?}: Checkt of de Scorpion levend is of niet.
	\item \textbf{try-boosting!}: Probeert de Schorpioen te versnellen zolang deze niet op cooldown is.
	\item \textbf{die!}: Veranderd de ``alive'' waarden naar \#f
\end{itemize}

%%%%%%%%%%%%%%%%%%%%%%%%%%%%%%%% VIEW %%%%%%%%%%%%%%%%%%%%%%%%%%%%%%%%%%%%%
\section{Views}
\label{view}
De Views zijn de ADTs verantwoordelijk voor het teken van het spel en het weergeven van de modellen op een visuele manier.

\subsection{View ADT}
\label{section:view}
De \texttt{\nameref{section:view}} is het centrale ADT voor het tekenlogica.
Deze ADT is verantwoordelijk voor het aanmaken van de canvas, lagen en nodige Views die de Player en de andere objecten in een level voorstellen.

