Dit document beschrijft het implementatie in fase 1 van het spel \texttt{Fire Ant} in scheme \textbf{r5rs}.
Het maakt deel uit van het opleidingsonderdeel "Programmeerproject 1".

\section{Spelbeschrijving}
In dit spel is het de bedoeling dat een speler een \textit{vuurmier} bestuurt,
die tracht zijn mierenkoningin te redden van schorpioenenleger.
Om dit te voltooien moet de speler een serie van levels, die toenemen in moeilijkheidsgraad,
oplossen zodat hij tot bij de koningin geraakt.
Een level bestaat uit een doolhof gevuld met puzzels, items en vijanden.
Enkel door deze puzzels op te lossen en de schorpioenen te vermijden, kan de speler het level oplossen.
Het spel eindigt dan tot dat alle levels opgelost zijn of tot dat de speler al zijn levens verloren heeft.

\section{Implementatie}
Voor het eerste fase zal het genoeg zijn om de elementaire elementen van het spel te implementeren.

Onder andere:
\begin{itemize}
	\item Het spel maakt een spelwereld met een doolhof waarin een vuurmier, die door de speler kan bestuurt worden, kan rond lopen.
	\item Het doolhof bevat ook (een) schorpioen(en) die een vaste pad aflegt.
	\item	De vuurmier moet bij aanraking van de schorpioen terug naar zijn start positie gaan.
	\item Verspreid doorheen het doolhof moeten er verschillenden eitjes geplaatst zijn,
		die vuurmier kan oprapen.
\end{itemize}

Het programma zal ook verdeelt worden uit
een controller (zal tussen de modellen en views moeten communiceren)
modellen (bezitten al de spellogica)
en de views (zorgt voor het tekenlogica van het spel)
