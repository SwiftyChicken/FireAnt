In deze sectie zullen alle ADTs en hun proceduren beschreven worden 
die in het programma voorkomen.
Deze ADTs zullen verdeelt worden in 3 categorieën:
\nameref{controller},
\nameref{model} en
\nameref{view}
%%%%%%%%%%%%%%%%%%%%%%%%%%%%% CONTROLLER %%%%%%%%%%%%%%%%%%%%%%%%%%%%%%%%%%
\section{Controller}
\label{controller}
De controller zal centraal zijn om de interacties tussen de modellen en views te beheren. 

\subsection{Game ADT}
\label{section:game}
Het \texttt{\nameref{section:game}} is het centrale ADT dat het spel opstart en beheert.
Deze is verantwoordelijk voor
het initialiseren van de \texttt{\nameref{section:player}},
het initialiseren en starten van de juiste \texttt{\nameref{section:level}},
het verwerken van de speler invoer
en het behandelen van het spellus.

\begin{table}[hbt]
\centering
\begin{tabular}{|ll|}
\hline
\rowcolor[HTML]{000000} 
{\color[HTML]{FFFFFF} \textbf{Naam}} & {\color[HTML]{FFFFFF} \textbf{Signatuur}} \\ \hline
new-game    & $($list $\rightarrow$ Game$)$                  \\ \hline
start       & $(\varnothing \rightarrow \varnothing)$        \\ \hline
next-level! & $(\varnothing \rightarrow \varnothing)$     	 \\ \hline
key-handler & $($symbol symbol $\rightarrow \varnothing)$    \\ \hline
game-loop   & $($number $\rightarrow \varnothing)$           \\ \hline
\end{tabular}
\caption{Operaties van het \texttt{\nameref{section:game}}}
\label{table:1}
\end{table}

Het \texttt{\nameref{section:game}} bevat de volgende operaties:

\begin{itemize}
	\item \textbf{new-game}: Deze operatie maakt een object van het \texttt{\nameref{section:game}} aan.
	\item \textbf{start}: Start een nieuw spel op.
		Eerst slaagt het alle namen van de levels (dit zijn csv bestanden) uit de "level" map op, in een lijst.
		Deze lijst gebruikt het om de eerste \texttt{\nameref{section:level}} object te maken.
		Vervolgens maakt het een nieuwe object van het \texttt{\nameref{section:player}} en het \texttt{\nameref{section:view}}.
		Tenslotte "start" het, de spellus en gebruikers invoer met behulp van de game-loop en key-handler procedures respectievelijk. 
	\item \textbf{next-level!}: Veranderd de huidige level (current-level) naar de volgende level in de levels lijst.
		De oude level wordt ook verwijderd uit de levels lijst.
		Het update ook de huidige level van de \texttt{\nameref{section:view}}.
	\item \textbf{key-handler}: Neemt twee argumenten op state en key.
		State is de status van de toets (key) op het huidige moment.
		Op dit moment ondersteunt de functie enkel de pijltjes toetsen ingedrukt.
		Wanneer dit gebeurt, beweegt het respectievelijk de speler als deze niet tegen een muur aanloopt.
	\item \textbf{game-loop}: Neemt een argument op ms (milliseconde voor vorige aanroep).
		Als de huidige level niet uitgespeeld is dan
		update het de huidige level en de view object.
		Daarnaast als de speler sterft, herleeft de level, de speler opnieuw.
		In het geval dat het huidige level uitgespeeld is voer het next-level procedure op.
\end{itemize}

%%%%%%%%%%%%%%%%%%%%%%%%%%%%%% MODELLEN %%%%%%%%%%%%%%%%%%%%%%%%%%%%%%%%%%%
\section{Modellen}
\label{model}
De Modellen zullen verantwoordelijk zijn voor het spellogica m.a.w. alle berekening achter de schermen.

\section{Level ADT}
\label{section:level}
Het \texttt{\nameref{section:level}} is het ADT dat een level opbouwt aan de hand van csv tekst bestand.
Deze is verantwoordelijk om
een csv bestand te lezen en zijn elementen ervan te interpreteren,
het onthouden en aanmaken van alle nodige objecten in de level (bv. schorpioenen, eiren, het doolhof, etc.),
de speler te laten herstarten,
het updaten van de elementen in de level,
het onthouden van de startpositie (spawn)
en te checken of het level is uitgespeeld.

\begin{table}[hbt]
\centering
\begin{tabular}{|ll|}
\hline
\rowcolor[HTML]{000000} 
{\color[HTML]{FFFFFF} \textbf{Naam}} & {\color[HTML]{FFFFFF} \textbf{Signatuur}} \\ \hline
new-level      & (Player string $\rightarrow$ Level)   \\ \hline
init           & (string $\rightarrow$ $\varnothing$)                      \\ \hline
get-maze       & ($\varnothing$ $\rightarrow$ Maze)                        \\ \hline
get-eggs       & ($\varnothing$ $\rightarrow$ list)                        \\ \hline
get-scorpions  & ($\varnothing$ $\rightarrow$ list)                        \\ \hline
get-updates    & ($\varnothing$ $\rightarrow$ list)                        \\ \hline
is-finished?   & (Player $\rightarrow$ boolean)                            \\ \hline
is-legal-move? & (Object\footnotemark{} symbol $\rightarrow$ boolean)      \\ \hline
update!        & ($\varnothing$ $\rightarrow$ $\varnothing)$               \\ \hline
respawn        & ($\varnothing$ $\rightarrow$ $\varnothing$)               \\ \hline
on-collision   & ((Object\footnotemark[\value{footnote}] $\rightarrow$ $\varnothing$) $\rightarrow$ $\varnothing$) \\ \hline
interpret!     & (string number number $\rightarrow$ $\varnothing$)        \\ \hline
\end{tabular}
\caption{Operaties van het \texttt{\nameref{section:level}}}
\label{table:level}
\end{table}


Het \texttt{\nameref{section:level}} bevat de volgende operaties:
\begin{itemize}
	\item \textbf{new-level}: Neemt een \texttt{\nameref{section:level}} en de naam van level csv bestand op als argument op.
		Deze procedure geeft een object van het type level terug.
		Deze level object bezit alle elementen die het level bezit.
	\item \textbf{init}: Deze procedure initialiseert het level door de element van het csv bestaand een na een door te geven aan het interpret! procedure
		met de x en y positie van de element in de csv bestand.
	\item \textbf{get-maze}: Geeft de \texttt{\nameref{section:maze}} object terug opgeslagen in de \texttt{\nameref{section:level}}.
	\item \textbf{get-eggs}: Geeft een lijst van \texttt{\nameref{section:egg}} objecten terug die opgeslagen zijn in de \texttt{\nameref{section:level}}.
	\item \textbf{get-scorpions}: Geeft een lijst van \texttt{\nameref{section:scorpion}} objecten terug die opgeslagen zijn in de \texttt{\nameref{section:level}}.
	\item \textbf{get-updates}: Geeft een lijst van objecten terug die in de huidige iteratie van de spellus geüpdatet zijn.
	\item \textbf{is-finished?}: Geeft aan of het spel is uitgespeeld door te kijken of de \texttt{\nameref{section:player}} helemaal onderaan de doolhof is geraakt.
	\item \textbf{is-legal-move?}: Neemt een Object (dat een \texttt{\nameref{section:positie}} bezit) en een direction ('up, 'down, 'left or 'right) als argument op.
		Het zoekt met met behulp van de positie van het Object naar zijn buur positie gelegen op de ``direction''.
		Tenslotte kijkt de functie of de gebuurt positie een muur is aan de hand van de \texttt{\nameref{section:maze}}.
	\item \textbf{update!}: De procedure update de ``updates'' lijst
		door eerst de nodige objecten te updaten (en dus ook toe te voegen aan de lijst).
		Vervolgens checkt het of er objecten zijn die botsen met het Player object 
		en zo ja, voeren ze hun eigen procedure uit 
		(bv. schorpioenen vermoorden de speler en eiren worden genomen/verdwijnen).
		Deze objecten worden dan ook toegevoegd aan de lijst.
	\item \textbf{respawn}: Deze procedure plaatst de Player terug op zijn start positie en geeft hem terug ``leven''.
	\item \textbf{on-collision}: Neemt een procedure (die wordt uitgevoerd als er een botsing met de Player gebeurt) 
		en een lijst van objecten die gecheckt moeten worden voor botsingen met de Player.
	\item \textbf{interpret!}: Deze procedure neemt een string
		(dat een element uit de csv bestand is)
		, een x en y positie (dat de positie van de string voorstelt in de csv bestand).
		Deze warden worden gebruikt om Objecten of de Maze van het \texttt{\nameref{section:level}}
\end{itemize}

\footnotetext{Object is een object dat een Positie ADT bezit.}

\section{Position ADT}
\label{section:positie}

De \texttt{\nameref{section:positie}} dit is een simpel ADT dat de positie in een level voorstelt.
isFinished?

\begin{table}[hbt]
\centering
\begin{tabular}{|ll|}
\hline
\rowcolor[HTML]{000000} 
{\color[HTML]{FFFFFF} \textbf{Naam}} & {\color[HTML]{FFFFFF} \textbf{Signatuur}} \\ \hline
newPositie                          & (number number Maze $\rightarrow$ $Positie\cup\{\#f\}$)                                       \\ \hline
move!                                & (number number $\rightarrow$ $\varnothing\cup\{\#f\}$)                                       \\ \hline
 eqPosistie?                                & (Positie $\rightarrow$ boolean)                                       \\ \hline
 <++>                                & (<++> $\rightarrow$ <++>)                                       \\ \hline
\end{tabular}
\caption{Operaties van het \texttt{\nameref{section:positie}}}
\label{table:positie}
\end{table}

Het \texttt{\nameref{section:positie}} bevat de volgende operaties:

\begin{itemize}
	\item \textbf{newPositie}: Deze operatie maakt een nieuw object van het \texttt{\nameref{section:positie}} aan. 
		Die twee numbers opneemt dat respectievelijk de x- en y-co\"ordinaat zijn in het meegegeven \texttt{\nameref{section:maze}}.
	\item \textbf{move!}: Verandert het positie van het \texttt{\nameref{section:positie}} als dat een geldige positie is in het \texttt{\nameref{section:maze}}.
	\item \textbf{eqPosistie?}: Controleert dat het meegegeven Positie een gelijkwaardige positie heeft.
		Dat wil zeggen dat het x- en y-co\"ordinaat een aan een gelijk zijn van elkaar. 
	\item \textbf{<++>}: <++>
\end{itemize}

\section{Egg ADT}
\label{section:egg}

Het \texttt{\nameref{section:egg}} stelt elk voorwerp voor dat kan opgeraapt  worden door een \texttt{Ant} (speler).
<++>

\begin{table}[hbt]
\centering
\begin{tabular}{|ll|}
\hline
\rowcolor[HTML]{000000} 
{\color[HTML]{FFFFFF} \textbf{Naam}} & {\color[HTML]{FFFFFF} \textbf{Signatuur}} \\ \hline
newItem                              & (Positie (any $rightarrow$ any) $\rightarrow$ Item) \\ \hline
isCollected?                         & ($\varnothing$ $\rightarrow$ boolean)               \\ \hline
update!                              & (Ant $\rightarrow$ $\varnothing$)                   \\ \hline
 <++>                                & (<++> $\rightarrow$ <++>)                           \\ \hline
\end{tabular}
\caption{Operaties van het \texttt{\nameref{section:egg}}}
\label{table:item}
\end{table}

Het \texttt{\nameref{section:egg}} bevat de volgende operaties:

\begin{itemize}
	\item \textbf{newItem}: Maakt een nieuw object van het \texttt{\nameref{section:egg}}.
		Het eerste argument dat de operatie neemt is een \texttt{\nameref{section:positie}} dat een positie in het maze (doolhof) aangeeft.
		Vervolgens het tweede argument, een functie dat wordt toegepast bij het oprapen van het \texttt{\nameref{section:egg}}.
	\item \textbf{isCollected?}: Check of het \texttt{\nameref{section:egg}} is opgeraapt of niet.
	\item \textbf{update!}: Deze operatie checkt of het Ant object op dezelfde positie heeft als het item.
		Als dat het geval is wordt het te toepassen effect toegepast door het functie dat werd meegegeven aan \textbf{newItem}.
		Anders doet het operatie niks.
\end{itemize}

\section{Character ADT}
\label{section:character}

Het \texttt{\nameref{section:character}} moet een elke beweegbare character voorstellen in het spel.
Deze is verantwoordelijk voor alle algemeen functies die een object van deze \texttt{ADT} bezit.

\begin{table}[hbt]
\centering
\begin{tabular}{|ll|}
\hline
\rowcolor[HTML]{000000} 
{\color[HTML]{FFFFFF} \textbf{Naam}} & {\color[HTML]{FFFFFF} \textbf{Signatuur}} \\ \hline
newCharacter                                 & (Positie $\rightarrow$ Character)                                       \\ \hline
\end{tabular}
\caption{Operaties van het \texttt{\nameref{section:character}}}
\label{table:character}
\end{table}

Het \texttt{\nameref{section:character}} bevat de volgende operaties:

\begin{itemize}
	\item \textbf{newCharacter}: Deze operatie maakt een object van het \texttt{\nameref{section:character}} aan.
	\item \textbf{<++>}: <++>
\end{itemize}

\section{Maze ADT}
\label{section:maze}

Het \texttt{\nameref{section:maze}} is verantwoordelijk om het spellogica van het doolhof te construeren.

\begin{table}[hbt]
\centering
\begin{tabular}{|ll|}
\hline
\rowcolor[HTML]{000000} 
{\color[HTML]{FFFFFF} \textbf{Naam}} & {\color[HTML]{FFFFFF} \textbf{Signatuur}} \\ \hline
newMaze                              & (string $\rightarrow$ Maze)               \\ \hline
isValidPosition?                      & (Positie $\rightarrow$ boolean)           \\ \hline
 <++>                                & (<++> $\rightarrow$ <++>)                 \\ \hline
\end{tabular}
\caption{Operaties van het \texttt{\nameref{section:maze}}}
\label{table:maze}
\end{table}

Het \texttt{\nameref{section:maze}} bevat de volgende operaties:

\begin{itemize}
	\item \textbf{newMaze}: Deze operatie maakt een object van het \texttt{\nameref{section:maze}} aan.
	\item \textbf{isValidPosition?}: Checkt als deze positie niet bezet is door een muur en dat het een bestaande positie is in het \texttt{\nameref{section:maze}}.
	\item \textbf{<++>}: <++>
\end{itemize}

\section{Views}
\label{view}
\section{View ADT}
\label{section:view}
\section{View\textunderscore Maze ADT}
\label{section:view_maze}

Het \texttt{\nameref{section:view_maze}} is verantwoordelijk om het \texttt{\nameref{section:maze}} voor te stellen in teken logica gedeelte van het spel.

\begin{table}[hbt]
\centering
\begin{tabular}{|ll|}
\hline
\rowcolor[HTML]{000000} 
{\color[HTML]{FFFFFF} \textbf{Naam}} & {\color[HTML]{FFFFFF} \textbf{Signatuur}} \\ \hline
newView\textunderscore Maze                                 & (Maze $\rightarrow$ View\textunderscore Maze)                                       \\ \hline
 <++>                                & (<++> $\rightarrow$ <++>)                 \\ \hline
\end{tabular}
\caption{Operaties van het \texttt{\nameref{section:view_maze}}}
\label{table:view_maze}
\end{table}

Het \texttt{\nameref{section:view_maze}} bevat de volgende operaties:

\begin{itemize}
	\item \textbf{++}: <++>
	\item \textbf{<++>}: <++>
\end{itemize}

\section{Scorpion ADT}
\label{section:scorpion}

<++>

\begin{table}[hbt]
\centering
\begin{tabular}{|ll|}
\hline
\rowcolor[HTML]{000000} 
{\color[HTML]{FFFFFF} \textbf{Naam}} & {\color[HTML]{FFFFFF} \textbf{Signatuur}} \\ \hline
<++>                                 & (<++> $\rightarrow$ <++>)                                       \\ \hline
\end{tabular}
\caption{Operaties van het \texttt{\nameref{section:scorpion}}}
\label{table:scorpion}
\end{table}

Het \texttt{\nameref{section:scorpion}} bevat de volgende operaties:

\begin{itemize}
	\item \textbf{<++>}: <++>
	\item \textbf{<++>}: <++>
\end{itemize}

\section{View\textunderscore Scorpion ADT}
\label{section:view_scorpion}

<++>

\begin{table}[hbt]
\centering
\begin{tabular}{|ll|}
\hline
\rowcolor[HTML]{000000} 
{\color[HTML]{FFFFFF} \textbf{Naam}} & {\color[HTML]{FFFFFF} \textbf{Signatuur}} \\ \hline
<++>                                 & (<++> $\rightarrow$ <++>)                                       \\ \hline
\end{tabular}
\caption{Operaties van het \texttt{\nameref{section:view_scorpion}}}
\label{table:view_scorpion}
\end{table}

Het \texttt{\nameref{section:view_scorpion}} bevat de volgende operaties:

\begin{itemize}
	\item \textbf{<++>}: <++>
	\item \textbf{<++>}: <++>
\end{itemize}

\section{Player ADT}
\label{section:player}

<++>

\begin{table}[hbt]
\centering
\begin{tabular}{|ll|}
\hline
\rowcolor[HTML]{000000} 
{\color[HTML]{FFFFFF} \textbf{Naam}} & {\color[HTML]{FFFFFF} \textbf{Signatuur}} \\ \hline
<++>                                 & (<++> $\rightarrow$ <++>)                                       \\ \hline
\end{tabular}
\caption{Operaties van het \texttt{\nameref{section:player}}}
\label{table:ant}
\end{table}

Het \texttt{\nameref{section:player}} bevat de volgende operaties:

\begin{itemize}
	\item \textbf{<++>}: <++>
	\item \textbf{<++>}: <++>
\end{itemize}

\section{View\textunderscore Player ADT}
\label{section:view_ant}

<++>

\begin{table}[hbt]
\centering
\begin{tabular}{|ll|}
\hline
\rowcolor[HTML]{000000} 
{\color[HTML]{FFFFFF} \textbf{Naam}} & {\color[HTML]{FFFFFF} \textbf{Signatuur}} \\ \hline
<++>                                 & (<++> $\rightarrow$ <++>)                                       \\ \hline
\end{tabular}
\caption{Operaties van het \texttt{\nameref{section:view_ant}}}
\label{table:view_ant}
\end{table}

Het \texttt{\nameref{section:view_ant}} bevat de volgende operaties:

\begin{itemize}
	\item \textbf{<++>}: <++>
	\item \textbf{<++>}: <++>
\end{itemize}

\section{Eggs ADT}
\label{section:eggs}

<++>

\begin{table}[hbt]
\centering
\begin{tabular}{|ll|}
\hline
\rowcolor[HTML]{000000} 
{\color[HTML]{FFFFFF} \textbf{Naam}} & {\color[HTML]{FFFFFF} \textbf{Signatuur}} \\ \hline
<++>                                 & (<++> $\rightarrow$ <++>)                                       \\ \hline
\end{tabular}
\caption{Operaties van het \texttt{\nameref{section:eggs}}}
\label{table:eggs}
\end{table}

Het \texttt{\nameref{section:eggs}} bevat de volgende operaties:

\begin{itemize}
	\item \textbf{<++>}: <++>
	\item \textbf{<++>}: <++>
\end{itemize}

\section{View\textunderscore Eggs ADT}
\label{section:view_eggs}

<++>

\begin{table}[hbt]
\centering
\begin{tabular}{|ll|}
\hline
\rowcolor[HTML]{000000} 
{\color[HTML]{FFFFFF} \textbf{Naam}} & {\color[HTML]{FFFFFF} \textbf{Signatuur}} \\ \hline
<++>                                 & (<++> $\rightarrow$ <++>)                                       \\ \hline
\end{tabular}
\caption{Operaties van het \texttt{\nameref{section:view_eggs}}}
\label{table:view_eggs}
\end{table}

Het \texttt{\nameref{section:view_eggs}} bevat de volgende operaties:

\begin{itemize}
	\item \textbf{<++>}: <++>
	\item \textbf{<++>}: <++>
\end{itemize}

