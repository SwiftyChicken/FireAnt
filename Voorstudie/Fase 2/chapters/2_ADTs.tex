In deze sectie zullen de verwachte nieuwe ADT's 
en nieuwe procedures besproken worden die toegevoegd zullen worden.

\section{Controller}
\label{controller}
\subsection{Game ADT}
\label{section:game}
Het \texttt{\nameref{section:game}} dat het spel opstart en de loop van het spel beheert,
zal er nu ook voorzorgen dat het spel herstart,
de volgende level laad 
of de level herstart.

Deze nieuwe functionaliteiten zullen vooral aan de hand van privé functies 
die in de het \texttt{\nameref{section:game}} zullen geïmplementeerd worden.
Dit is omdat enkel het \texttt{\nameref{section:game}} verantwoordelijk is voor het beheren van het spel verloop 
en dus deze functionaliteiten daar buiten niet worden gebruikt.
Deze functionaliteiten zullen toegevoegd worden aan de hand van de volgende procedures:

\begin{table}[hbt]
\centering
\begin{tabular}{|ll|}
\hline
\rowcolor[HTML]{000000} 
{\color[HTML]{FFFFFF} \textbf{Naam}} & {\color[HTML]{FFFFFF} \textbf{Signatuur}} \\ \hline
next-level!    & $(\varnothing \rightarrow \varnothing)$     	 \\ \hline
restart-level! & $(\varnothing \rightarrow \varnothing)$     	 \\ \hline
\end{tabular}
\caption{Operaties van het \texttt{\nameref{section:game}}}
\label{table:1}
\end{table}

\begin{itemize}
	\item \textbf{next-level!}: Veranderd de huidige level (current-level) naar de volgende level in de levels lijst.
		Deze lijst zal door de Game onthouden worden.
		De oude level wordt ook verwijderd uit de levels lijst.
	\item \textbf{restart-level!}: Vernieuwt de current-level variabelen
		door het Level ADT opnieuw te maken.
\end{itemize}

\section{Modellen}
\label{model}

\subsection{Position ADT}
\label{section:positie}
De \texttt{\nameref{section:positie}} zal nu ook verantwoordelijk zijn voor het snelheid van een object.
Hiermee bedoelen we de snelheid waarmee een object van een positie naar het volgende verplaatst.
Dit gaan we implementeren met locale variable van de \texttt{\nameref{section:positie}}, ``speed''.
De volgende procedures zullen ook gebruikt worden om deze locale variabel te behandelen:


\begin{table}[hbt]
\centering
\begin{tabular}{|ll|}
\hline
\rowcolor[HTML]{000000} 
{\color[HTML]{FFFFFF} \textbf{Naam}} & {\color[HTML]{FFFFFF} \textbf{Signatuur}} \\ \hline
get-speed        & ($\varnothing$ $\rightarrow$ number)                        \\ \hline
set-speed!        & (number $\rightarrow$ $\varnothing$)                        \\ \hline
\end{tabular}
\caption{Operaties van het \texttt{\nameref{section:positie}}}
\label{table:positie}
\end{table}

\begin{itemize}
	\item \textbf{get-speed}: Geeft de snelheid van het object terug.
		Deze is standaard 0,17.
	\item \textbf{set-speed!}: Verandert de snelheid door de meegegeven snelheid.
\end{itemize}

\subsection{Item ADT}
\label{section:item}
Het \texttt{Egg ADT} zal nu veranderd worden in het \texttt{\nameref{section:item}}.
Deze ADT zal bovenop de functionaliteiten van het \texttt{Egg ADT} ook nog zijn score waarde en item soort kunnen onthouden.

\begin{table}[hbt]
\centering
\begin{tabular}{|ll|}
\hline
\rowcolor[HTML]{000000} 
{\color[HTML]{FFFFFF} \textbf{Naam}} & {\color[HTML]{FFFFFF} \textbf{Signatuur}} \\ \hline
new-egg                              & (number Positie $\rightarrow$ Egg) \\ \hline
get-item-type         & ($\varnothing$ $\rightarrow$ Position)                        \\ \hline
take!                              & (Player $\rightarrow$ $\varnothing$)                   \\ \hline
\end{tabular}
\caption{Operaties van het \texttt{\nameref{section:item}}}
\label{table:item}
\end{table}

\begin{itemize}
	\item \textbf{new-egg}: De new-egg porcedure neemt nu ook een nummer op.
		Deze nummer bepaal welke soort item wordt gecreëerd
		welke score waarde deze heeft.
	\item \textbf{get-item-type}: Geeft het soort van het item terug.
		Deze soort zal belangrijk zijn voor het tekenlogica.
	\item \textbf{take!}: Veranderd taken waarden naar \#t.
		en zal de Player de gepaste scoren bijgeven.
		Deze functie zou ook gebruikt kunnen worden voor de Player een power-up te geven.
\end{itemize}

\subsection{Player ADT}
\label{section:player}
De \texttt{\nameref{section:player}} zal nu ook een score moeten bewaren
en deze kunnen aanpassen wanneer nodig.
Dit kunnen we doen met lokale score variabel die van 0 begint.
De volgende procedures zullen ook nog toegevoegd moeten worden:

\begin{table}[hbt]
\centering
\begin{tabular}{|ll|}
\hline
\rowcolor[HTML]{000000} 
{\color[HTML]{FFFFFF} \textbf{Naam}} & {\color[HTML]{FFFFFF} \textbf{Signatuur}} \\ \hline
get-score  & ($\varnothing$ $\rightarrow$ number)                        \\ \hline
add-score! & (number $\rightarrow$ $\varnothing$)                        \\ \hline
die!          & ($\varnothing$ $\rightarrow$ $\varnothing$)                   \\ \hline
\end{tabular}
\caption{Operaties van het \texttt{\nameref{section:player}}}
\label{table:ant}
\end{table}

\begin{itemize}
	\item \textbf{get-score}: Geeft huidige score terug.
	\item \textbf{add-score!}: Voegt punten toe aan de score aan de hand van de meegegeven waarde.
	\item \textbf{die!}: Deze procedure zal de score waarden ook terug naar 0 zetten als de speler geen levens meer heeft.
\end{itemize}

\subsection{Scorpion ADT}
\label{section:scorpion}
De \texttt{\nameref{section:scorpion}} zal nu een nieuwe type schorpioen moet ondersteunen (een schorpioen zonder vaste pad).
Dit zou kunnen geïmplementeerd worden door een lege lijst mee te geven als pad.
Het schorpioen moet nu ook nog op een willekeurige moment versnellen.
Dit zou geïmplementeerd kunnen worden door de volgende procedures.

\begin{table}[hbt]
\centering
\begin{tabular}{|ll|}
\hline
\rowcolor[HTML]{000000} 
{\color[HTML]{FFFFFF} \textbf{Naam}} & {\color[HTML]{FFFFFF} \textbf{Signatuur}} \\ \hline
update!                                 & ($\varnothing$ $\rightarrow$ $\varnothing$)                                       \\ \hline
try-boost!                                 & ($\varnothing$ $\rightarrow$ $\varnothing$)                                       \\ \hline
is-boosted?  & ($\varnothing$ $\rightarrow$ boolean)                        \\ \hline
is-random?  & ($\varnothing$ $\rightarrow$ boolean)                        \\ \hline
\end{tabular}
\caption{Operaties van het \texttt{\nameref{section:scorpion}}}
\label{table:scorpion}
\end{table}

\begin{itemize}
	\item \textbf{update!}: Update zal nu een willekeurige richting toepassen als de ``path'' lijst leeg is.
	\item \textbf{try-boost!}: De procedure heeft een bepaalde kans om de boost variabel te veranderen naar (not boost),
		alleen als de schorpioen niet meer beweegt naar een nieuwe richting.
	\item \textbf{is-boosted?}: Geeft aan of de schorpioen versneld is of niet.
	\item \textbf{is-random?}: Geeft aan of de schorpioen een willekeurige ``path'' volgt.
\end{itemize}
