Dit document beschrijft het implementatie van het spelletjes \texttt{Fire Ant} in scheme \textbf{r5rs}.
Het maakt deel uit van het opleidingsonderdeel "Programmeerproject 1".

\section{Spelbeschrijving}

In dit spel is het de bedoeling dat een speler een \textit{vuurmier} bestuurt, die tracht zijn mierenkoningin te redden van schorpioenenleger.
Om dit te voltooien moet de speler een serie van levels, die toenemen in moeilijkheidsgraad, oplossen zodat hij tot bij de koningin kan geraken.
Zo een level is een doolhof dat gevuld is met (verschillende soorten) puzzels en schorpioenen die de doorgangen bewaken.
Enkel door deze puzzels op te lossen en de schorpioenen te vermijden, kan de speler het level oplossen.
Het spel eindigt dan tot dat alle levels opgelost zijn of tot dat de speler al zijn levens kwijt is geraakt.

\section{Implementatie}

Voor het spel te implementeren zal er gewerkt worden in twee fases.
In deze document zal enkel de implementatie van het eerste fase besporken worden.
Het doel van deze fase zal het implementeren van de elementaire componenten en handelingen zijn.\\
\\
Om het spel te implementeren zal er gebruikt gemaakt worden van een spellus.
Deze spellus moet de invoer van de speler behandelen, de spellogica en de elementen in het spel updaten bij elke iteratie/frame.\\
zelf worden geupdate telkens als deze element door het spel wordt geupdate.
Omdat dit enkel het eerste fase is zullen er enkel de volgende elementaire functionaliteiten ge\"implementeerd worden:

\begin{itemize}
	\item Een spelwereld dat een verzameling zal moeten bevatten van doolhoven (levels).
	\item Een doolhof dat de plaatsing van de muren, in- en uitgang bepaalt.
	\item Een vuurmier dat kan bestuurt worden op basis van de gebruikersinvoer.
	\item Een schorpioen die een vaste pad bezit en kan zien of hij de vuurmier (speler) aanraakt.
	\item Eitjes die een vast positie hebben in het doolhof en kunnen check of ze door de vuurmier (speler) zijn opgeraapt.
\end{itemize}

Het tekenlogica voor elk van deze elementen zal respectievelijk zelf door het element bijgehouden en geupdate worden.
Met andere woorden, elke elment dat op het spel weergegeven moet worden zal, naast het spellogica, ook een andere element bijhouden dat het tekenlogica zal voorstellen.

