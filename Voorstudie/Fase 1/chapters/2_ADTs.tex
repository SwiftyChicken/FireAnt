In deze sectie zullen alle ADTs beschreven dat men verwacht te gebruiken, om de elementaire functionaliteiten in fase 1 van \texttt{Fire Ant} te implementeren.

\section{Game ADT}
\label{section:game}

Het \texttt{\nameref{section:game}} is het centrale ADT det het spel op te start.
Deze is verantwoordelijk om het initialisatie en onthouden van het juiste level en deze te updaten of te herstarten wanneer nodig.
De \texttt{\nameref{section:game}} is ook verantwoordelijk om naar de invoer te luisteren en deze door te geven aan de \texttt{\nameref{section:level}}.
Het luisteren zal gebeuren aan de hand van een spellus. 

\begin{table}[hbt]
\centering
\begin{tabular}{|ll|}
\hline
\rowcolor[HTML]{000000} 
{\color[HTML]{FFFFFF} \textbf{Naam}} & {\color[HTML]{FFFFFF} \textbf{Signatuur}} \\ \hline
newGame                             & $(list \rightarrow Game)$                  \\ \hline
start!                               & $(\varnothing \rightarrow \varnothing)$    \\ \hline
<??>                                 & <??>                                       \\ \hline
\end{tabular}
\caption{Operaties van het \texttt{\nameref{section:game}}}
\label{table:1}
\end{table}

Het \texttt{\nameref{section:game}} bevat de volgende operaties:

\begin{itemize}
	\item \textbf{newGame}: Dit is de aanmaak operatie van het \texttt{\nameref{section:game}}.
		Voor objecten van \texttt{\nameref{section:game}} te maken, moet er een lijst meegegeven worden van strings. Deze strings zijn de namen van tekst bestanden die zullen gebruikt worden door het \texttt{\nameref{section:level}}.
	\item \textbf{start!}: Deze operatie start het spel op. 
		Wanneer deze operatie uitgevoerd wordt zal er een venster aangemaakt worden.
		Ook de operatie om een nieuw ant en het corresponderende level worden aangeroepen in het spel lus van het spel.
	\item \textbf{<??>}: <??>
\end{itemize}

\section{Level ADT}
\label{section:level}

Het \texttt{\nameref{section:level}} is het ADT verantwoordelijk om het level te vormen aan de hand van een tekst bestand en alles op zijn juiste plaats te zetten.
Deze is verantwoordelijk om het level, characters en items te initialiseren.
<??>

\begin{table}[hbt]
\centering
\begin{tabular}{|ll|}
\hline
\rowcolor[HTML]{000000} 
{\color[HTML]{FFFFFF} \textbf{Naam}} & {\color[HTML]{FFFFFF} \textbf{Signatuur}} \\ \hline
newLevel                            & (string Ant $\rightarrow$ Level)   \\ \hline
update!                              & (Ant $\rightarrow$ $\varnothing)$    \\ \hline
isFinished?                          & ($\varnothing$ $\rightarrow$ boolean) \\ \hline
 <??>                                & (<??> $\rightarrow$ <??>)             \\ \hline
\end{tabular}
\caption{Operaties van het \texttt{\nameref{section:level}}}
\label{table:level}
\end{table}

Het \texttt{\nameref{section:level}} bevat de volgende operaties:

\begin{itemize}
	\item \textbf{newLevel}: Dit is het aanmaak operatie van het \texttt{\nameref{section:level}}. Voor een object van het \texttt{\nameref{section:level}} aan te maken moet er een string en een object van het \texttt{\nameref{section:ant}} meegegeven worden.
		Het string geeft het naam van het bestand weer dat door het \texttt{\nameref{section:maze}}, die in het \texttt{\nameref{section:level}} zit, zal gebruikt worden.
		Het object van het \texttt{\nameref{section:ant}} is belangrijk voor het positioneren van de speler in het maze.<??>
	\item \textbf{update!}: Deze operatie is verantwoordelijk om alle objecten van het \texttt{\nameref{section:character}} en het \texttt{\nameref{section:item}}, die respectievelijk opgeslagen zijn in een lijst, te updaten.
		Het updaten van elk object gebeurt door hun respectievelijke \textit{\textbf{update!}} operatie aan te roepen.
	\item \textbf{isFinished}: Deze operatie checkt of het level gedaan is of niet.
	\item \textbf{<??>}: <??>
\end{itemize}

\section{Position ADT}
\label{section:positie}

De \texttt{\nameref{section:positie}} dit is een simpel ADT dat de positie in een level voorstelt.
isFinished?

\begin{table}[hbt]
\centering
\begin{tabular}{|ll|}
\hline
\rowcolor[HTML]{000000} 
{\color[HTML]{FFFFFF} \textbf{Naam}} & {\color[HTML]{FFFFFF} \textbf{Signatuur}} \\ \hline
newPositie                          & (number number Maze $\rightarrow$ $Positie\cup\{\#f\}$)                                       \\ \hline
move!                                & (number number $\rightarrow$ $\varnothing\cup\{\#f\}$)                                       \\ \hline
 eqPosistie?                                & (Positie $\rightarrow$ boolean)                                       \\ \hline
 <??>                                & (<??> $\rightarrow$ <??>)                                       \\ \hline
\end{tabular}
\caption{Operaties van het \texttt{\nameref{section:positie}}}
\label{table:positie}
\end{table}

Het \texttt{\nameref{section:positie}} bevat de volgende operaties:

\begin{itemize}
	\item \textbf{newPositie}: Deze operatie maakt een nieuw object van het \texttt{\nameref{section:positie}} aan. 
		Die twee numbers opneemt dat respectievelijk de x- en y-co\"ordinaat zijn in het meegegeven \texttt{\nameref{section:maze}}.
	\item \textbf{move!}: Verandert het positie van het \texttt{\nameref{section:positie}} als dat een geldige positie is in het \texttt{\nameref{section:maze}}.
	\item \textbf{eqPosistie?}: Controleert dat het meegegeven Positie een gelijkwaardige positie heeft.
		Dat wil zeggen dat het x- en y-co\"ordinaat een aan een gelijk zijn van elkaar. 
	\item \textbf{<??>}: <??>
\end{itemize}

\section{Item ADT}
\label{section:item}

Het \texttt{\nameref{section:item}} stelt elk voorwerp voor dat kan opgeraapt  worden door een \texttt{Ant} (speler).
<??>

\begin{table}[hbt]
\centering
\begin{tabular}{|ll|}
\hline
\rowcolor[HTML]{000000} 
{\color[HTML]{FFFFFF} \textbf{Naam}} & {\color[HTML]{FFFFFF} \textbf{Signatuur}} \\ \hline
newItem                              & (Positie (any $rightarrow$ any) $\rightarrow$ Item) \\ \hline
isCollected?                         & ($\varnothing$ $\rightarrow$ boolean)               \\ \hline
update!                              & (Ant $\rightarrow$ $\varnothing$)                   \\ \hline
 <??>                                & (<??> $\rightarrow$ <??>)                           \\ \hline
\end{tabular}
\caption{Operaties van het \texttt{\nameref{section:item}}}
\label{table:item}
\end{table}

Het \texttt{\nameref{section:item}} bevat de volgende operaties:

\begin{itemize}
	\item \textbf{newItem}: Maakt een nieuw object van het \texttt{\nameref{section:item}}.
		Het eerste argument dat de operatie neemt is een \texttt{\nameref{section:positie}} dat een positie in het maze (doolhof) aangeeft.
		Vervolgens het tweede argument, een functie dat wordt toegepast bij het oprapen van het \texttt{\nameref{section:item}}.
	\item \textbf{isCollected?}: Check of het \texttt{\nameref{section:item}} is opgeraapt of niet.
	\item \textbf{update!}: Deze operatie checkt of het Ant object op dezelfde positie heeft als het item.
		Als dat het geval is wordt het te toepassen effect toegepast door het functie dat werd meegegeven aan \textbf{newItem}.
		Anders doet het operatie niks.
\end{itemize}

\section{Character ADT}
\label{section:character}

Het \texttt{\nameref{section:character}} moet een elke beweegbare character voorstellen in het spel.
Deze is verantwoordelijk voor alle algemeen functies die een object van deze \texttt{ADT} bezit.

\begin{table}[hbt]
\centering
\begin{tabular}{|ll|}
\hline
\rowcolor[HTML]{000000} 
{\color[HTML]{FFFFFF} \textbf{Naam}} & {\color[HTML]{FFFFFF} \textbf{Signatuur}} \\ \hline
newCharacter                                 & (Positie $\rightarrow$ Character)                                       \\ \hline
\end{tabular}
\caption{Operaties van het \texttt{\nameref{section:character}}}
\label{table:character}
\end{table}

Het \texttt{\nameref{section:character}} bevat de volgende operaties:

\begin{itemize}
	\item \textbf{newCharacter}: Deze operatie maakt een object van het \texttt{\nameref{section:character}} aan.
	\item \textbf{<??>}: <??>
\end{itemize}

\section{Maze ADT}
\label{section:maze}

Het \texttt{\nameref{section:maze}} is verantwoordelijk om het spellogica van het doolhof te construeren.

\begin{table}[hbt]
\centering
\begin{tabular}{|ll|}
\hline
\rowcolor[HTML]{000000} 
{\color[HTML]{FFFFFF} \textbf{Naam}} & {\color[HTML]{FFFFFF} \textbf{Signatuur}} \\ \hline
newMaze                              & (string $\rightarrow$ Maze)               \\ \hline
isValidPosition?                      & (Positie $\rightarrow$ boolean)           \\ \hline
 <??>                                & (<??> $\rightarrow$ <??>)                 \\ \hline
\end{tabular}
\caption{Operaties van het \texttt{\nameref{section:maze}}}
\label{table:maze}
\end{table}

Het \texttt{\nameref{section:maze}} bevat de volgende operaties:

\begin{itemize}
	\item \textbf{newMaze}: Deze operatie maakt een object van het \texttt{\nameref{section:maze}} aan.
	\item \textbf{isValidPosition?}: Checkt als deze positie niet bezet is door een muur en dat het een bestaande positie is in het \texttt{\nameref{section:maze}}.
	\item \textbf{<??>}: <??>
\end{itemize}

\section{View\textunderscore Maze ADT}
\label{section:view_maze}

Het \texttt{\nameref{section:view_maze}} is verantwoordelijk om het \texttt{\nameref{section:maze}} voor te stellen in teken logica gedeelte van het spel.

\begin{table}[hbt]
\centering
\begin{tabular}{|ll|}
\hline
\rowcolor[HTML]{000000} 
{\color[HTML]{FFFFFF} \textbf{Naam}} & {\color[HTML]{FFFFFF} \textbf{Signatuur}} \\ \hline
newView\textunderscore Maze                                 & (Maze $\rightarrow$ View\textunderscore Maze)                                       \\ \hline
 <??>                                & (<??> $\rightarrow$ <??>)                 \\ \hline
\end{tabular}
\caption{Operaties van het \texttt{\nameref{section:view_maze}}}
\label{table:view_maze}
\end{table}

Het \texttt{\nameref{section:view_maze}} bevat de volgende operaties:

\begin{itemize}
	\item \textbf{++}: <++>
	\item \textbf{<++>}: <++>
\end{itemize}

\section{Scorpion ADT}
\label{section:scorpion}

<++>

\begin{table}[hbt]
\centering
\begin{tabular}{|ll|}
\hline
\rowcolor[HTML]{000000} 
{\color[HTML]{FFFFFF} \textbf{Naam}} & {\color[HTML]{FFFFFF} \textbf{Signatuur}} \\ \hline
<++>                                 & (<++> $\rightarrow$ <++>)                                       \\ \hline
\end{tabular}
\caption{Operaties van het \texttt{\nameref{section:scorpion}}}
\label{table:scorpion}
\end{table}

Het \texttt{\nameref{section:scorpion}} bevat de volgende operaties:

\begin{itemize}
	\item \textbf{<++>}: <++>
	\item \textbf{<++>}: <++>
\end{itemize}

\section{View\textunderscore Scorpion ADT}
\label{section:view_scorpion}

<++>

\begin{table}[hbt]
\centering
\begin{tabular}{|ll|}
\hline
\rowcolor[HTML]{000000} 
{\color[HTML]{FFFFFF} \textbf{Naam}} & {\color[HTML]{FFFFFF} \textbf{Signatuur}} \\ \hline
<++>                                 & (<++> $\rightarrow$ <++>)                                       \\ \hline
\end{tabular}
\caption{Operaties van het \texttt{\nameref{section:view_scorpion}}}
\label{table:view_scorpion}
\end{table}

Het \texttt{\nameref{section:view_scorpion}} bevat de volgende operaties:

\begin{itemize}
	\item \textbf{<++>}: <++>
	\item \textbf{<++>}: <++>
\end{itemize}

\section{Ant ADT}
\label{section:ant}

<++>

\begin{table}[hbt]
\centering
\begin{tabular}{|ll|}
\hline
\rowcolor[HTML]{000000} 
{\color[HTML]{FFFFFF} \textbf{Naam}} & {\color[HTML]{FFFFFF} \textbf{Signatuur}} \\ \hline
<++>                                 & (<++> $\rightarrow$ <++>)                                       \\ \hline
\end{tabular}
\caption{Operaties van het \texttt{\nameref{section:ant}}}
\label{table:ant}
\end{table}

Het \texttt{\nameref{section:ant}} bevat de volgende operaties:

\begin{itemize}
	\item \textbf{<++>}: <++>
	\item \textbf{<++>}: <++>
\end{itemize}

\section{View\textunderscore Ant ADT}
\label{section:view_ant}

<++>

\begin{table}[hbt]
\centering
\begin{tabular}{|ll|}
\hline
\rowcolor[HTML]{000000} 
{\color[HTML]{FFFFFF} \textbf{Naam}} & {\color[HTML]{FFFFFF} \textbf{Signatuur}} \\ \hline
<++>                                 & (<++> $\rightarrow$ <++>)                                       \\ \hline
\end{tabular}
\caption{Operaties van het \texttt{\nameref{section:view_ant}}}
\label{table:view_ant}
\end{table}

Het \texttt{\nameref{section:view_ant}} bevat de volgende operaties:

\begin{itemize}
	\item \textbf{<++>}: <++>
	\item \textbf{<++>}: <++>
\end{itemize}

\section{Eggs ADT}
\label{section:eggs}

<++>

\begin{table}[hbt]
\centering
\begin{tabular}{|ll|}
\hline
\rowcolor[HTML]{000000} 
{\color[HTML]{FFFFFF} \textbf{Naam}} & {\color[HTML]{FFFFFF} \textbf{Signatuur}} \\ \hline
<++>                                 & (<++> $\rightarrow$ <++>)                                       \\ \hline
\end{tabular}
\caption{Operaties van het \texttt{\nameref{section:eggs}}}
\label{table:eggs}
\end{table}

Het \texttt{\nameref{section:eggs}} bevat de volgende operaties:

\begin{itemize}
	\item \textbf{<++>}: <++>
	\item \textbf{<++>}: <++>
\end{itemize}

\section{View\textunderscore Eggs ADT}
\label{section:view_eggs}

<++>

\begin{table}[hbt]
\centering
\begin{tabular}{|ll|}
\hline
\rowcolor[HTML]{000000} 
{\color[HTML]{FFFFFF} \textbf{Naam}} & {\color[HTML]{FFFFFF} \textbf{Signatuur}} \\ \hline
<++>                                 & (<++> $\rightarrow$ <++>)                                       \\ \hline
\end{tabular}
\caption{Operaties van het \texttt{\nameref{section:view_eggs}}}
\label{table:view_eggs}
\end{table}

Het \texttt{\nameref{section:view_eggs}} bevat de volgende operaties:

\begin{itemize}
	\item \textbf{<++>}: <++>
	\item \textbf{<++>}: <++>
\end{itemize}

